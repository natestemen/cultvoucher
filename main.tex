\documentclass{cultvoucher}

\addbibresource{pubs.bib}
\addtocategory{papers}{backgrounds,p20paper,mitiq,per}

\begin{document}

\begin{adjustwidth}{\dimexpr-\marginparsep-\marginparwidth}{}
	\begin{center}
		\textsc{\texttt{\HUGE Nathaniel T. Stemen}}

		\email{nate@stemen.email} \separator{}\, Seattle, WA
	\end{center}
\end{adjustwidth}

\section{Employment}

\begin{entry}{Member of Technical Staff}{Unitary Fund}{Mar 2022--}
	\item Lead developer, product manager, and team coordinator for the open-source Python package \texttt{mitiq} (175,000+ downloads, 110+ citations).
		Drive both technical innovation and strategic roadmap development to enhance quantum error mitigation.
		Manage contributors, oversee project planning, and foster collaboration within a distributed, open-source team.
	\item Developed software to test hypotheses for publications and integrated novel quantum error mitigation techniques (classical shadows, and quantum subspace expansion) from research literature into \texttt{mitiq}, bridging theory and practice.
	\item Manage a thriving open-source community by organizing weekly technical talks, facilitating community calls, and conducting user outreach to inform teams development across multiple projects.
    \item Directed unitaryHACK 2023, overseeing event planning, issue curation, and participant engagement.
		Coordinated efforts across 70 hackers to close 99 issues across the quantum open-source ecosystem, distributing over \$11,000 in rewards to contributors.
\end{entry}

\begin{entry}{Software Developer}{Overleaf}{2017--2021}
    \item Improved \LaTeX{} autocomplete using statistical analysis of open-source documents, enhancing user experience for 300,000+ daily users.
    \item Maintained and optimized large \keyword{Rails} and \keyword{Node} applications through bug fixes, performance improvements, and feature delivery.
    \item Monitored and supported data migration from \keyword{PostgresQL} to \keyword{MongoDB}, ensuring data integrity throughout the process.
\end{entry}

\begin{entry}{Summer Researcher}{New York University}{2016}
	\item Used \keyword{Python} to numerically solve nonlinear Schr\"{o}dinger equations modeling electromagnetic pulse propagation in nonlinear media.
\end{entry}

\begin{entry}{Summer Researcher}{Yale University (PROSPECT Experiment)}{2014 \& 2015}
    \item Built an optical simulation in \keyword{C++} to optimize detector design and study light collection and uniformity.
    \item Implemented pulse-shape discrimination techniques in \keyword{Python} to improve neutrino event selection.
\end{entry}

\section{Education}

\begin{entry}{University of Waterloo}{MMath in Applied Mathematics}{2020--2022}
	\item Thesis: \textit{Quantum Circuit Compilation from the Ground Up} advised by Joel Wallman
\end{entry}

\begin{entry}{New York University}{B.Sc.\ in Mathematics and Physics}{2013--2017}
	\item Thesis: \textit{An Investigation of $\mathcal{Q}$-Balls} advised by Luciano Medina
\end{entry}

\section{Publications} % TODO the spacing after headings is larger than other parts of doc
\printbib{papers}

\section{Teaching}

\secitem{Fundamentals of University Teaching}{University of Waterloo}{2020--2022}
\begin{itemize}
	\item Completed program designed to help graduate students learn evidence-based strategies for teaching through workshops and practice teaching sessions.
\end{itemize}

\secitem{Mathematics Teacher}{{\small NYU Metro Center College Prep Academy}}{2015--2017}
\begin{itemize}
	\item Independently planned and taught Pre-Calculus course for high school students.
	\item Facilitated numerous extra-curricular math courses of 30 students as a class assistant by providing additional guidance to students.
\end{itemize}

\section{Service}
\secitem{QED-C mentor}{}{2023--}
\secitem{Equity, Diversity and Inclusion Committee}{University of Waterloo; IQC}{2021--2022}
\secitem{Strategic Plan Implementation Working Group}{University of Waterloo}{2021}
% \begin{itemize}
% 	\item Working with the mathematics department to attract and retain people of high potential and accomplishment as well as foster student, staff, and faculty wellbeing.
% \end{itemize}


\section{Coninuing Education}
\secitem{CSE 534: Quantum information and computation}{University of Washington (audit)}{Sep--Dec 2024}
\secitem{Quantum Machine Learning Workshop}{\href{https://www.cmc.ca/qscitech-quantumbc-virtual-workshop-2022/}{QSciTech-QuantumBC}}{Jan--Feb 2022}
\secitem{Presenting Data and Information}{Edward Tufte}{Nov 2019}


\section{Tools}
\secitem{Languages}{}{}
\begin{itemize}
	\item Python, JavaScript, SQL, Ruby, bash, HTML
	\item English (native), Mandarin Chinese (beginner)
\end{itemize}
\secitem{Software}{}{}
\begin{itemize}
	\item git/GitHub, AWS, docker, Linux, MacOS
\end{itemize}

\end{document}
