\documentclass{cultvoucher}

\addbibresource{pubs.bib}
\addtocategory{papers}{backgrounds,p20paper,mitiq,per}

\begin{document}

\begin{adjustwidth}{\dimexpr-\marginparsep-\marginparwidth}{}
	\begin{center}
		\textsc{\texttt{\HUGE Nathaniel T. Stemen}}

		\email{nate@stemen.email} \separator{}\, Seattle, WA
	\end{center}
\end{adjustwidth}

\section{Summary}
\hfill\break
Research engineer experienced in quantum error mitigation, circuit compilation, and translating quantum research into practical tools for NISQ devices.
Lead developer of \texttt{mitiq}, a widely-used open-source Python library for error mitigation.
Passionate about building open-source tools that enable the next generation of quantum computers.

\section{Education}

\begin{entry}{University of Waterloo}{MMath in Applied Mathematics}{2020--2022}
	\item Thesis: \textit{Quantum Circuit Compilation from the Ground Up} advised by Joel Wallman
\end{entry}

\begin{entry}{New York University}{B.Sc.\ in Mathematics and Physics}{2013--2017}
	\item Thesis: \textit{An Investigation of $\mathcal{Q}$-Balls} advised by Luciano Medina
\end{entry}

\section{Employment}

\begin{entry}{Member of Technical Staff}{Unitary Foundation}{Mar 2022--}
	\item Lead developer for the open-source \keyword{Python} package \texttt{mitiq} (250,000+ downloads, 130+ citations).
		Drive both technical innovation and strategic roadmap development to enhance performance and adoption of quantum error mitigation.
	\item Developed a calibration module to match error mitigation techniques and parameters to the characteristics of users' quantum hardware.
	\item Delivered talks on quantum error mitigation and resilience at conferences and seminars including PyData, SciPy, QuSoft, and IPN Mexico, as well as a tutorial at IEEE QCE.
	\item Led community-building efforts in quantum computing, including directing unitaryHACK 2023 (70 hackers, 99 issues closed, \$11,000+ distributed) and serving as chair organizer for the Workshop on Error Resilience in Quantum Computing (\href{https://werq.shop}{WERQ.SHOP}), convening researchers to shape the roadmap for error-resilient quantum systems over the next 5 years.
\end{entry}

\begin{entry}{Software Developer}{Overleaf}{2017--2021}
    \item Improved \LaTeX{} autocomplete using statistical analysis of open-source documents, enhancing user experience for 300,000+ daily users.
    \item Maintained and optimized large \keyword{Rails} and \keyword{Node} applications through bug fixes, performance improvements, and feature delivery.
    \item Monitored and supported data migration from \keyword{PostgresQL} to \keyword{MongoDB}.
\end{entry}

\begin{entry}{Summer Researcher}{New York University}{2016}
	\item Used \keyword{Python} to numerically solve nonlinear Schr\"{o}dinger equations modeling electromagnetic pulse propagation in nonlinear media.
\end{entry}

\begin{entry}{Summer Researcher}{Yale University (PROSPECT Experiment)}{2014 \& 2015}
    \item Built an optical simulation in \keyword{C++} to optimize detector design and study light collection and uniformity.
    \item Implemented pulse-shape discrimination techniques in \keyword{Python} to improve neutrino event selection.
\end{entry}

\section{Publications} % TODO the spacing after headings is larger than other parts of doc
\printbib{papers}

\section{Teaching}

\secitem{Fundamentals of University Teaching}{University of Waterloo}{2020--2022}
\begin{itemize}
	\item Completed program designed to help graduate students learn evidence-based strategies for teaching through workshops and practice teaching sessions.
\end{itemize}

\secitem{Mathematics Teacher}{{\small NYU Metro Center College Prep Academy}}{2015--2017}
\begin{itemize}
	\item Independently planned and taught Pre-Calculus course for high school students.
	\item Facilitated numerous extra-curricular math courses of 30 students as a class assistant by providing additional guidance to students.
\end{itemize}

\section{Service}
\secitem{IEEE QCE 2025 Workshop organizer}{Quantum Software 2.1}{2025}
\secitem{WERQSHOP Chair Organizer}{\url{https://werq.shop}}{2025}
\secitem{SciPy 2025 Reviewer}{}{2025}
\secitem{QED-C mentor}{}{2023--2024}
\secitem{Equity, Diversity and Inclusion Committee}{University of Waterloo; IQC}{2021--2022}
\secitem{Strategic Plan Implementation Working Group}{University of Waterloo}{2021}
% \begin{itemize}
% 	\item Working with the mathematics department to attract and retain people of high potential and accomplishment as well as foster student, staff, and faculty wellbeing.
% \end{itemize}


\section{Continuing\\Education}
\secitem{CSE 599C: Quantum Learning Theory}{University of Washington (audit)}{Jan--Mar 2025}
\secitem{CSE 534: Quantum info. and computation}{University of Washington (audit)}{Sep--Dec 2024}
\secitem{Quantum Machine Learning Workshop}{\href{https://www.cmc.ca/qscitech-quantumbc-virtual-workshop-2022/}{QSciTech-QuantumBC}}{Jan--Feb 2022}
\secitem{Presenting Data and Information}{Edward Tufte}{Nov 2019}


\section{Tools}
\secitem{Languages}{}{}
\begin{itemize}
	\item Python, JavaScript, SQL, Ruby, bash
	% \item English (native), Mandarin Chinese (beginner)
\end{itemize}
\secitem{Software}{}{}
\begin{itemize}
	\item git/GitHub, docker, Linux, MacOS, \LaTeX{}
\end{itemize}
\secitem{Quantum}{}{}
\begin{itemize}
	\item SDKs: Cirq, Qiskit, pyQuil, Qibo
\end{itemize}

\end{document}
