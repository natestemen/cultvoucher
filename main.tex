\documentclass{cultvoucher}

\usepackage[margin=0.6in]{geometry}
\usepackage{setspace} % not sure if needed
\usepackage{cultvoucher}
% \usepackage{fancyhdr}
\definecolor{OliveGreen}{rgb}{0.33, 0.42, 0.18}
\usepackage[colorlinks=true,urlcolor=OliveGreen]{hyperref}
\usepackage{stackengine}
\usepackage[T1]{fontenc}
\usepackage[protrusion=true,expansion=true]{microtype}
\usepackage{newpxtext}
\usepackage[document]{ragged2e}
\usepackage{paracol}


\renewcommand{\labelitemi}{--}


\bibliography{pubs}
\addtocategory{papers}{backgrounds,p20paper,czts}
\addtocategory{presentations}{ceu15,gcurs16}
\addtocategory{curpapers}{vortices}



\begin{document}

\begin{flushleft}
    {\Huge{\textsc{\texttt{Nathaniel T. Stemen}}}}
    
    {\href{https://natestemen.com}{\texttt{natestemen.com}} |
    \href{mailto:nathanielstemen@gmail.com}{\nolinkurl{nathanielstemen@gmail.com}}} |
    \href{tel:12038154690}{\texttt{+1\,203\,815\,4690}}
\end{flushleft}

\vspace{-10pt}

\columnratio{0.7}
\begin{paracol}{2}

\section{Work \& Research Experience}

\lhsentry{Software Development/Data Science Intern}
         {\href{https://www.overleaf.com/}{Overleaf}}
         {Jul--Nov 2017}
\begin{tightitemize}
    \item Built framework using \keyword{pandas}, \keyword{sklearn}, and
          \keyword{scipy} to parse, mung, and analyze \TeX{} files to obtain
          data for new Overleaf autocomplete feature.
    \item Using common patterns found in \TeX{} files, extended Overleaf's
          editor to support context-aware autocompletion by using a
          recommendation system.
\end{tightitemize}

\lhsentry{Undergraduate Researcher}
         {New York University}
         {May 2016--May 2017}
\begin{tightitemize}
	\item Studied nonlinear Schr\"{o}dinger equations modeling transmission of
          short electromagnetic pulses in nonlinear media under Professor
          \href{https://www.sites.google.com/a/nyu.edu/luciano-medina/}{Luciano Medina}.
	\item Numerically computed solutions using \keyword{numpy} to nonlinear
          PDE's and analytically proved existence of solutions.
	\item Built package to numerically solve functional minimization problems.
\end{tightitemize}

\lhsentry{Summer Researcher}
         {Yale University (\href{http://prospect.yale.edu/}{PROSPECT Experiment})}
         {Summer \stackanchor{2014}{2015}}
\begin{tightitemize}
	\item Completed R\&D for detector that will perform eV-scale sterile
          neutrino search and measure the antineutrino spectrum from the
          nuclear reactor at Oak Ridge National Laboratory under Professor
          \href{http://heegerlab.yale.edu/karsten-heeger}{Karsten Heeger}.
	\item Built optical simulation using SLitrani, and \keyword{C++} to model
          prototype detector to study light collection, detector uniformity,
          and optimize light guide shape.
	\item Surveyed, and implemented pulse-shape discrimination methods in
          \keyword{Python} to determine optimal method for PROSPECT Experiment.
\end{tightitemize}


\switchcolumn


\section{Education}
\vspace{-0.9\topsep}
\rhsentry{New York University}{2013--2017}{B.S. in Physics and Mathematics}
\begin{tightitemize}
    \item Thesis in Mathematics: \textit{An \mbox{Investigation} of
          $\mathcal{Q}$-Balls}
    \item Dean's List
    \item University Honors Scholar
\end{tightitemize}
\vspace{-2\topsep}

\section{Skills}\vspace{-0.9\topsep}
\rhsentry{Languages}
         {most$\rightarrow$least}
         {Python, [Coffee|Java]Script, SQL, Ruby, Mathematica, MATLAB}

\rhsentry{Tools}
         {}
         {git/Github, Bash, \LaTeX{}, Chebfun, HTML, CSS}

% \rhsentry{Interests}{}{Cycling, (skate\!|\!snow)boarding, climbing, typography, books.}
\vspace{-2\topsep}

\section{Affiliations}\vspace{-0.9\topsep}
\rhsentry{Sigma Pi Sigma}{2015--}{Physics Honors Society}

\rhsentry{American Physical Society}{2014--}{}

\rhsentry{NYU Society of Undergraduate Physicists}
         {2014--17}
         {President, Vice-President, and Secretary.}
\vspace{-2\topsep}

\end{paracol}
\vspace{-0.3\topsep}

\raggedright
\columnratio{0.6}
\begin{paracol}{2}

\section{Publications}

\printbib{curpapers}
\printbib{papers}


\switchcolumn


\section{Talks \& Presentations}
\printbib{presentations}

\end{paracol}
\vspace{-2\topsep}

% \section{Professional Writing}
% \subsection{Overleaf Blog}
% \begin{itemize}
% \setlength{\itemsep}{0pt}\setlength{\parskip}{0pt}\setlength{\parsep}{0pt}
%     \item \href{https://www.overleaf.com/blog/523-a-data-driven-approach-to-latex-autocomplete}{A Data-Driven Approach to \LaTeX{} Autocomplete}
% \end{itemize}
% \subsection{Mostly Math (personal blog)}
% \begin{itemize}
% \setlength{\itemsep}{0pt}\setlength{\parskip}{0pt}\setlength{\parsep}{0pt}
%     \item \href{https://natestemen.github.io/html/blog.html#}{Some Mathematics of Shuffling Cards}
% \end{itemize}

\section{Teaching Experience}
\lhsentry{Mathematics Teacher}{NYU Metro Center College Prep Academy}{Jun--Aug 2016}
\begin{tightitemize}
	\item Independently planned and taught Pre-Calculus class for high school
          students.
	\item Created and graded in class worksheets, quizzes, and homework.
	\item Used the Moore Method to guide students through advanced topics and
          introduce the idea of rigor in mathematics.
% 	\item Manager: Patricia Ryan-Canedo (\href{mailto:par3@nyu.edu}{\texttt{par3@nyu.edu}})
\end{tightitemize}
\vspace{\topsep}

\lhsentry{Mathematics Tutor}{NYU Metro Center College Prep Academy}{Oct 2015--May 2017}
\begin{tightitemize}
	\item Facilitated numerous extra-curricular math courses of 30 students as
          a class assistant by providing additional guidance to students.
	\item Specialized in tutoring Algebra II and Pre-Calculus, focusing on
          strong fundamentals.
% 	\item Manager: Patricia Ryan-Canedo (\href{mailto:par3@nyu.edu}{\texttt{par3@nyu.edu}})
\end{tightitemize}
% \datedsubsection{Orientation Leader, NYU Tandon School of Engineering}{Aug--Sep \stackanchor{2014}{2015}}
% \begin{itemize}
% 	 \setlength{\itemsep}{1pt}\setlength{\parskip}{0pt}\setlength{\parsep}{0pt}
% 	\item Worked in teams of two guiding groups of 25 new students through NYU orientation week.
% 	\item Organized, coordinated, and facilitated events encouraging new students to socialize and discover NYU.
	
% \end{itemize}


\end{document}
